\section{C++}
\subsection{Максимальный уровень бинарного дерева}

\begin{lstlisting}
// A queue based C++ program to find maximum sum
#include <iostream>
#include <queue>

using namespace std ;

/* A binary tree node has data, pointer to left child
 and a pointer to right child */
struct Node
{
    int data ;
    struct Node * left, * right ;
};

// Function to find the maximum sum of a level in tree
// using level order traversal
int maxLevelSum(struct Node * root)
{
    // Base case
    if (root == NULL)
        return 0;
    
    // Initialize result
    int result = root->data;
    
    // Do Level order traversal keeping track of number
    // of nodes at every level.
    queue<Node*> q;
    q.push(root);
    while (!q.empty())
    {
        // Get the size of queue when the level order
        // traversal for one level finishes
        int count = q.size() ;
        
        // Iterate for all the nodes in the queue currently
        int sum = 0;
        while (count--)
        {
            // Dequeue an node from queue
            Node *temp = q.front();
            q.pop();
            
            // Add this node's value to current sum.
            sum = sum + temp->data;
            
            // Enqueue left and right children of
            // dequeued node
            if (temp->left != NULL)
                q.push(temp->left);
            if (temp->right != NULL)
                q.push(temp->right);
        }
        
        // Update the maximum node count value
        result = max(sum, result);
    }
    
    return result;
}

/* Helper function that allocates a new node with the
 given data and NULL left and right pointers. */
struct Node * newNode(int data)
{
    struct Node * node = new Node;
    node->data = data;
    node->left = node->right = NULL;
    return (node);
}

int main()
{
    struct Node *root = newNode(1);
    root->left        = newNode(2);
    root->right       = newNode(3);
    root->left->left  = newNode(4);
    root->left->right = newNode(5);
    root->right->right = newNode(8);
    root->right->right->left  = newNode(6);
    root->right->right->right  = newNode(7);
    
    
    cout << "Maximum level sum is "
    << maxLevelSum(root) << endl;
    return 0;
}	
\end{lstlisting}

\subsection{Все перестановки массива}
\begin{lstlisting}
#include <string.h>
#include <iostream>
using namespace std;

/* Function to swap values at two pointers */
void swap(int *x, int *y)
{
    int temp;
    temp = *x;
    *x = *y;
    *y = temp;
}

void prnt(int a[], int n=3){
    for (int i=0;i<n;i++){
        cout<<a[i];
    }
     cout<<'\n';
}
int* permute(int a[], int r)
{
    int i;
    if (r==0){

        return a;
    }
    else
    {
        for (i = 0; i < r; i++)
        {
            swap(a[i], a[r-1]);
            prnt(permute(a, r-1));
            swap(a[i], a[r-1]);
        }
    }
    return a;
}

int main()
{
    int a[3] = {1,2,3};
    int r=3;
    permute(a, r);
    
}
\end{lstlisting}
\subsection{Поиск в ширину}
\begin{lstlisting}
vector < vector<int> > g; // graph
int n;
int s;

queue<int> q;
q.push (s);
vector<bool> used (n);
vector<int> d (n), p (n);
used[s] = true;
p[s] = -1;
while (!q.empty()) {
	int v = q.front();
	q.pop();
	for (size_t i=0; i<g[v].size(); ++i) {
		int to = g[v][i];
		if (!used[to]) {
			used[to] = true;
			q.push (to);
			d[to] = d[v] + 1;
			p[to] = v;
		}
	}
}

if (!used[to])
	cout << "No path!";
else {
	vector<int> path;
	for (int v=to; v!=-1; v=p[v])
		path.push_back (v);
	reverse (path.begin(), path.end());
	cout << "Path: ";
	for (size_t i=0; i<path.size(); ++i)
		cout << path[i] + 1 << " ";
}
\end{lstlisting}

\subsection{Число простых делителей}
\begin{lstlisting}
	int count_or_prime_dividers(int x){
    int res = 0;
    for (int i=2;i<=x;i++){
        if (x % i ==0){
            cout<<i<<'\n';
            res+=1;
        }
        while (x%i ==0){
            x = x/ i;
        }
    }
    return res;
}
\end{lstlisting}
