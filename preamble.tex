\documentclass[a5paper,12pt]{extarticle}
\special{papersize=148mm,210mm}% it is A5 paper size, I got from Wikipedia.
\usepackage{fontspec} % XeTeX
\usepackage{xunicode} % Unicode для XeTeX
\usepackage{xltxtra}  % Верхние и нижние индексы
\usepackage{pdfpages} % Вставка PDF

% Русский язык
\usepackage{polyglossia}
\setdefaultlanguage{russian}
\setotherlanguage{english}
\usepackage{indentfirst} % Красная строка после заголовка
\renewcommand{\baselinestretch}{1.1}

% Шрифты, xelatex
\defaultfontfeatures{Ligatures=TeX}
\setmainfont{CMU Serif}
\newfontfamily{\cyrillicfonttt}{Menlo}

% Геометрия страницы
\usepackage{geometry}
\geometry{left=2cm}
\geometry{right=1cm}
\geometry{top=2cm}
\geometry{bottom=2cm}

% Отключение номеров формул
\usepackage{mathtools}  
\mathtoolsset{showonlyrefs}  

% Списки
\usepackage{enumitem}
\setlist{nolistsep} % Нет отступов между пунктами списка
\renewcommand{\labelitemi}{•}
\usepackage{float} % Расширенное управление плавающими объектами

%Формулы
\usepackage{amsmath}
\usepackage[makeroom]{cancel}

%Код
\usepackage{listings} 
\lstset{
	language=C++,
    basicstyle=\scriptsize\linespread{0.94}\ttfamily, % Размер и тип шрифта
    breaklines=true, % Перенос строк
    tabsize=2, % Размер табуляции
    literate={--}{{-{}-}}2 % Корректно отображать двойной дефис
}

% Пути к каталогам с изображениями
\usepackage{graphicx} % Вставка картинок и дополнений
\graphicspath{{illustrations/}}