\documentclass[a5paper,12pt]{extarticle}
\special{papersize=148mm,210mm}% it is A5 paper size, I got from Wikipedia.
\usepackage{fontspec} % XeTeX
\usepackage{xunicode} % Unicode для XeTeX
\usepackage{xltxtra}  % Верхние и нижние индексы
\usepackage{pdfpages} % Вставка PDF

% Русский язык
\usepackage{polyglossia}
\setdefaultlanguage{russian}
\setotherlanguage{english}
\usepackage{indentfirst} % Красная строка после заголовка
\renewcommand{\baselinestretch}{1.1}

% Шрифты, xelatex
\defaultfontfeatures{Ligatures=TeX}
\setmainfont{CMU Serif}
\newfontfamily{\cyrillicfonttt}{Menlo}

% Геометрия страницы
\usepackage{geometry}
\geometry{left=2cm}
\geometry{right=1cm}
\geometry{top=2cm}
\geometry{bottom=2cm}

% Отключение номеров формул
\usepackage{mathtools}  
\mathtoolsset{showonlyrefs}  

% Списки
\usepackage{enumitem}
\setlist{nolistsep} % Нет отступов между пунктами списка
\renewcommand{\labelitemi}{•}
\usepackage{float} % Расширенное управление плавающими объектами

%Формулы
\usepackage{amsmath}
\usepackage[makeroom]{cancel}

%Код
\usepackage{listings} 
\lstset{
	language=C++,
    basicstyle=\scriptsize\linespread{0.94}\ttfamily, % Размер и тип шрифта
    breaklines=true, % Перенос строк
    tabsize=2, % Размер табуляции
    literate={--}{{-{}-}}2 % Корректно отображать двойной дефис
}

% Пути к каталогам с изображениями
\usepackage{graphicx} % Вставка картинок и дополнений
\graphicspath{{illustrations/}}

\begin{document}

\section{Теория вероятностей}

\textbf{Схемой Бернулли}  называется последовательность независимых испытаний, в каждом из которых возможны лишь два исхода — «успех» и «неудача», при этом успех в каждом испытании происходит с одной и той же вероятностью  $p \in (0, 1)$, а неудача — с вероятностью $q = 1 - p$.

\textbf{Формула Бернулли} вероятность того, что событие наступит ровно $k$ раз при $n$ испытаниях
\begin{equation}
	P_{k,n}=C_n^k\cdot p^k \cdot q^{n-k}
\end{equation}

\subsection{Локальная теорема Муавра — Лапласа}

Если в схеме Бернулли $n$ стремится к бесконечности, то  
\begin{equation}
	P(a \leq\frac{\mu - np}{\sqrt{npq}} \leq b) 	\approx \int_{a}^{b} \frac{1}{\sqrt{2\pi}} e^{-\frac{x^2}{2}} dx = \Phi(b) - \Phi(a)
\end{equation}
где $\mu$ -- количество успехов, $\int_a^b$ -интеграл Лапласа. Рекомендуется использовать при $n > 100$ и $\mu > 20$. 

На заметку (для схемы Б.):
\begin{itemize}
	\item $M[\mu] = np$ -- математическое ожидание
	\item $D[\mu] = npq$ -- дисперсия
	\item $ \sigma = \sqrt{D[\mu]} = \sqrt{npq}$ -- среднеквадратичное отклонение
\end{itemize}
\vspace{1em}
И факты про интеграл Лапласа :
\begin{enumerate}
	\item $\Phi(-x) = - \Phi(x)$ \textit{важно!}
	\item $\Phi(x)\approx  0.5$, если $x > 5$
\end{enumerate}

\subsection{Центральная предельная теорема}
Утверждает о том, что сумма одинаково распределённых случайных и независимых случайных велечин, имеет распределение, близкое к нормальному.

Пусть $\xi_1 \dots \xi_n$ - последовательность случайных величин, $S_n = \sum_{i=1}^n \xi_i$, тогда:
\begin{equation}
	\frac{S_n-nM[\xi_k]}{\sqrt{nD[\xi_k]}} \rightarrow N(0.1)
\end{equation}

Cледствие:
\begin{equation}
	P(a \leq \frac{S_n-nM[\xi_k]}{\sqrt{nD[\xi_k]}} \leq b) \approx \Phi(b) - \Phi(a)
\end{equation}

\subsection{Неравенства Чебышева}
\begin{enumerate}
	\item $P(|\xi| \ge \varepsilon) \le \cfrac{M[\xi]}{\varepsilon}$ 
	\item $P(|\xi - M[\xi]| \ge \varepsilon) \le \cfrac{D[\xi]}{\varepsilon^2}$
	\item $P(|\xi - M[\xi]| < \varepsilon) \ge 1 - \cfrac{D[\xi]}{\varepsilon^2}$
\end{enumerate}


\subsection{Локальные предельные теоремы}
\begin{itemize}
	\item Пуассона: $P(\mu = k) \approx \cfrac{\lambda^k}{k!}e^{-\lambda}$, где $\lambda = np$. Использовать, когда $np \le 10$
	\item Муавра-Лапласа: $P(\mu = k)\approx \cfrac{\Phi(x)}{\sigma}$, где $x = \cfrac{k-np}{\sigma}$
\end{itemize}
\section{Дискретная математика}

\section{C++}
%TODO корректный алгоритм перестановок
\section{Алгебра и геометрия}
\end{document}